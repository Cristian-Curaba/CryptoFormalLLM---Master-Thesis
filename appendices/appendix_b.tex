\section{Extracts From The Evaluation Results}
This appendix section will report some extracts from the evaluation results.

\subsection{Struggling to follow instructions rules}
\label{Ex:struggling_to_follow_instructions}
\noindent \textbf{Input:}
\begin{small}\begin{spverbatim}
[...]
## Output Formatting Guidelines:   

- File Overwriting (always in Agent_execution folder):
    ```shell
    execute: cat << 'EOF' > Agent_execution/[filename]
    [file content]
    EOF
    ```
[...]
\end{spverbatim}\end{small}

The following output is taken from Task 1.2 in the execution of Protocol 2 with core model Claude 3 Haiku. The output format required is not respected (missing "shell" after "```").

\noindent \textbf{Output:}
\begin{small}\begin{spverbatim}
[...]
2. Saving the file:
   ```
   cat <<  'EOF' > Agent_execution/protocol.anb
   Protocol Protocol2:

   Knowledge:
    [...]
   EOF
   ```
\end{spverbatim}\end{small}

The following output is from Task 1.3 in executing Protocol 3 with core model Claude 3 Haiku. The output format required is not respected (missing "```shell").

\noindent \textbf{Output:}
\begin{small}\begin{spverbatim}
    Understood. I will follow a methodical approach to compare and finalize
    the protocol in Tamarin syntax.
    execute: cat << 'EOF' > Agent_execution/final_protocol_and_property.spthy
    theory Protocol3 
    begin
    [...]
    EOF
\end{spverbatim}\end{small}


\subsection{Fixing Vulnerability}
\label{Ex:fixing_vulnerability}
Instead of following input instructions, the model plans to fix the protocol vulnerability. This is a behaviour observed in Claude 3.5 Sonnet and Claude 3 Opus.

\noindent \textbf{Input:}
\begin{small}\begin{spverbatim}
[...]
Whenever an attack trace is displayed, you have successfully completed your goal.
Just copy it in `Agent_execution/TamarinTrace.txt` with the following command and,
after that, write  **next_step**:
```shell
execute: cat << 'EOF' > Agent_execution/TamarinTrace.txt
[copy here the attack trace]
EOF
```                                                                                            
[...]

==============================================================================                          
Attack trace for freshness:                                                                             
Symmetric_key_setup : [ Fr( ~symK ) ] --[  ]->
[ !Key( $A, $B, ~symK ) ]                                                                    
Init_Knowledge : [ !Key( $A, $B, ~symK ), Fr( ~M.1 ) ] --[  ]->
[ St_init_A( $A, $B, ~M.1, ~symK ), St_init_B( $A, $B, ~symK ) ]            
[...]
St_msg3_A( $A, $B, ~M, ~N, ~symK ) ]  
==============================================================================
[...]              
\end{spverbatim}\end{small}

The following output is from Task 2.2 executing Protocol 1 with core model Claude 3 opus.
Here, instead of writing the Tamarin-produced attack trace in the file, it tries to address the vulnerability.

\noindent \textbf{Output:}
\begin{spverbatim}
Here is a summary of the analysis and next steps: 
```summary                                                                                           
To address this, the protocol needs to be modified to prevent replay attacks and ensure the freshness of messages. Some potential approaches:                                                                                        
1. Include a fresh nonce from B in the second message that A must echo back in the third message. This would tie the sessions together.                                                                                              
2. Have both parties contribute fresh randomness that is mixed into the encryption key for each session.                                                                                                                             
3. Use timestamps or sequence numbers to ensure messages cannot be replayed from old sessions.                             
```
**Next step**                                                                                                                     
\end{spverbatim}
\vspace{0.5cm}
A similar behaviour can be drawn also from Protocol 2 execution (Claude 3 opus).

\noindent \textbf{Output:}
\begin{spverbatim}
Here is the analysis of the Tamarin feedback and the steps to address the identified issues:

Analysis of Tamarin Feedback:
1. Tamarin successfully loaded, translated, and analyzed the Protocol2 theory. 
2. The secrecy lemma was falsified, meaning Tamarin found an attack trace where the secret message M was revealed to the attacker.
3. The attack trace shows:
   - The attacker sends the constant 'zero' to B
   - B responds with the shared key XORed with 'zero', revealing the key to the attacker 
   - A sends a fresh nonce Na to the attacker
   - The attacker sends back Na XORed with the compromised shared key
   - A encrypts the secret M XORed with Na using the compromised key and sends it
   - The attacker can decrypt this to obtain M XORed with Na, and since it has Na, it can recover M

To fix the protocol and prevent this attack:
1. In rule B_1, B should not respond to an arbitrary message from the network by XORing it with the shared key. This directly reveals the key.
2. The protocol should be redesigned with proper authentication of the messages, for example using a challenge-response with the nonces.
3. The encryption of M should not be done by XORing it with the nonce, as this allows the attacker to cancel out the nonce if it can learn it.

Here are the changes made to the theory file to implement these fixes:

```shell
execute: cat << 'EOF' > Agent_execution/final_protocol_and_property.spthy
theory Protocol2
[...]
\end{spverbatim}